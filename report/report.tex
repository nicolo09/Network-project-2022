\documentclass[a4paper,12pt]{report}

\usepackage{alltt, fancyvrb, url}
\usepackage{graphicx}
\usepackage[utf8]{inputenc}
\usepackage{float}
\usepackage{hyperref}
\usepackage{adjustbox}
\usepackage{siunitx}
\usepackage{tabularx}
\sisetup{group-separator={\text{\space}}}

% Questo commentalo se vuoi scrivere in inglese.
\usepackage[italian]{babel}

\usepackage[italian]{cleveref}

\title{Relazione di progetto \\``Programmazione di reti''\\Traccia 1 - Progetto DRONI}

\author{Nicolò Guerra \and
Emma Leonardi \and
Filippo Casadei 
}

\begin{document}

\maketitle

\tableofcontents

\chapter{Analisi dei requisiti}

Si ipotizzi di dover gestire una rete di consegne di piccoli pacchi a domicilio tramite l’utilizzo di droni. La rete è
composta da:

\begin{itemize}
    \item un client da cui l’operatore assegna a ciascun drone l’indirizzo di consegna del pacco
    \item tre droni
    \item un gateway che provvede ad effettuare il relay dei messaggi verso i droni e a fare da concentratore per raccogliere i messaggi provenienti dai droni ed inviarli al client
\end{itemize}

Il client deve poter inviare l’indirizzo di consegna del pacco ad un drone, solo dopo che lo stesso drone si sia
presentato al gateway come «disponibile». La connessione tra droni e gateway è di tipo UDP mentre la
connessione tra client e gateway è di tipo TCP.
Ogni drone ha un suo indirizzo IPv4, e cosi le due interfacce del gateway e il client.
Sulla console del client si deve poter inserire l’indirizzo di consegna del pacco e l’identificativo (o anche l’ip
address) del drone che dovrà effettuare la consegna. Tali informazioni devono essere inviate al gateway che
provvederà ad inviare l’informazione sulla destinazione del pacco al drone incaricato. Tali informazioni devono
essere visibili sulla console del gateway.
Infine sulla console del Drone dovrà essere visibile l’indirizzo di destinazione del pacchetto. Il drone dovrà quindi
inviare al gateway un messaggio di avvenuta consegna del pacchetto (ipotizzare un tempo randomico in un
intervallo a scelta) e rendersi nuovamente disponibile.
Sulla console del gateway devono comparire tutti i messaggi in transito con sorgente e destinatario.
I 3 droni hanno un indirizzamento appartenente ad una rete di Classe C del tipo 192.168.1.0/24
Il Gateway ha due interfacce di rete: quella verso i droni il cui IP Address appartiene alla stessa network dei dispositivi mentre l’interfaccia verso il client ha indirizzo
ip appartenente alla classe 10.10.10.0/24, classe a cui appartiene anche l’IP address del gateway.
Si realizzi un emulatore Python che sfruttando il concetto dei socket visti in laboratorio consenta di simulare, utilizzando l’interfaccia di loopback del proprio PC, il
comportamento di questo sistema.
Si devono simulare le conessioni UDP dei device verso il Gateway e la connessione TCP del Gateway verso il Client. Inoltre indicare la dimensione dei buffer utilizzati
su ciascun canale trasmissivo, il tempo impiegato per trasmettere il pacchetto UDP ed il tempo impiegato per trasmettere il pacchetto TCP.

\chapter{Descrizione del funzionamento}



\end{document}